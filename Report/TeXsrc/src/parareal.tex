\section{Parareal}

We would like to solve the autonomous ordinary differential equation:
\begin{equation*}
  \begin{cases}
    u'(t) = f(u), & t \in [t_0, t_f] \\
    u(t_0) = u_0
  \end{cases}
\end{equation*}
where $f : \mathbb{R}^d \to \mathbb{R}^d$ and $u: \mathbb{R} \to \mathbb{R}^d$.

Parareal is an iterative scheme to approximate $u$ for the above, which can be
derived from the concept of \textit{multiple shooting} methods. The idea behind
these methods are to turn the problem into a nonlinear optimization problem,
which then can be solved via Newton-Raphson. \cite{gandervandewalle}

Take our time domain $[t_0, t_f]$ and partition it into $N$ pieces $0 = t_0 <
\dots < t_N = t_f$. Looking specifically at the interval $[t_n, t_{n+1}]$, we
pose the new ODE:
\[
  \begin{cases}
    u'_n = f(u_n), & t \in [t_n, t_{n+1}] \\
    u(t_n) = U_n
  \end{cases}
\]
where $U_n$ is such that $U_n - u_{n-1}(t_n, U_{n-1}) = 0$, i.e. the initial
condition satisfies the solution to the previous time slice. The conditions on
$U_i$ form a nonlinear system $F(U) = 0$, which we can approximate using
Newton-Raphson, so we recieve the iteration $U^{k+1} = U^k -
J_F^{-1}(U^k)F(U^k)$. As it turns out, as seen in Gander and Vandewalle
\cite{gandervandewalle}, that we can actually change this into the form:
\[
  \begin{cases}
    U_0^{k+1} = u_0 \\
    U_{n+1}^{k+1} = u_n(t_{n+1}, U_n^k) + \frac{\partial u_n}{\partial
    U_n}(t_{n+1}, U_n^k)(U_n^{k+1} - U_n^k)
  \end{cases}
\]
If we were to then, call $u_n(t_{n+1}, U_n^k) = \fine(t_{n+1}, t_n, U_n^k)$
where $F$ is some near truth integrator and were to approximate the second
term with another integrator $\coarse(t_{n+1},t_n,U_n^{k+1}) -
\coarse(t_{n+1},t_n, U_n^k)$, then we would recieve the iteration:
\[
  \begin{cases}
    U_0^{k+1} = u_0 \\
    U_{n+1}^{k+1} = \fine(t_{n+1}, t_n, U_n^k) + \coarse(t_{n+1},t_n,U_n^{k+1})
    - \coarse(t_{n+1},t_n, U_n^k)
  \end{cases}
\]
This is precisely what we call parareal iteration. Alternatively, you can think
about this as the predictor-corrector scheme:
\[
  \begin{cases}
    U_0^{k+1} = u_0 \\
    U_{n+1}^{k+1} = \coarse(t_{n+1},t_n,U_n^{k+1}) + \left(\fine(t_{n+1}, t_n,
    U_n^k) - \coarse(t_{n+1},t_n, U_n^k)\right)
  \end{cases}
\]
Regardless, we notice that this iteration, which is an approximation to our
solution as $k \to \infty$, has a term which is decoupled from the current
iteration step, $\fine$. Furthermore, it satisfies a \textit{first same as last}
property with the coarse integrators $\coarse$. The idea behind the parareal
method is to take advantage of both of these properties.
