\section{Conclusion}

In conclusion, we find that Parareal is indeed capable of providing significant
speedup to our general numerical methods for ordinary differential equations.
however, it's difficult to determine what's the correct choice of $\fine$ and
$\coarse$, since the stability function is not easy to analyze in general, and
because there's an unfomfortable optimization problem between wanting $\coarse$
to be as accurate as possible, while making sure that the ratio $T_f/T_g$ is as
large as possible. In addition, for systems with strong requirements on $\Delta
t$, such as for the heat equation, we might have no choice but to make $\Delta
t$ small, and thus making $T_g$ large with respect to $T_f$. This implies that
for these system, there might be better approaches to parallelism.

Some points to personally reflect on this project:
\begin{itemize}
  \item I should not have used OpenMP to implement parareal. In the end, I was
    limited in the number of processors I could test on due to it, and the
    pipelined efficient implementation was just mimicing MPI-like behavior
    anyway. If I had to do this again, I would use a combination of MPI and
    OpenMP, MPI would split the domain over different compute nodes, and OpenMP
    would internally handle the parallel computation of the fine integrations
    assigned to that node.
\item I ran out of time to test implicit integrators, and how their stability
  regions functioned. I found it difficult to implement them, since the way I
  had written my code isolated the actual $f$ of the ODE $u' = f(t,u)$, and
  thus if I wanted implicit solvers, I had to solve a nonlinear optimization
  problem, which was too much trouble.
\item I wanted to look at generalizations and improvements of parareal, such as
  PFASST, which takes a spectral deffered corrections approach to the iteration,
  and another which adapts it to a multigrid scheme, but I ran out of time.
\end{itemize}
